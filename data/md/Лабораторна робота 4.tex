I'm working on a Лабораторна робота №4. "Прийняття рішення щодо оптимальності
плану виготовлення розмірів деталей згідно оцінки керованості процесом"
Мета роботи: навчити студентів здійснювати побудову гістограм,
емпіричних кривих та кривої нормального розподілу.

problem related to Using the
Завдання роботи: на основі обчислень та побудованих гістограм,
емпіричних кривих та кривої нормального розподілу, прийняти рішення щодо
оптимальності плану виготовлення розмірів деталей.
Теоретичні відомості

and
Для виявлення закономірностей похибок, які виникають в параметрах
процесу, що досліджується використовують різні способи, які ґрунтуються на
методах математичної статистики. Виявлення цих закономірностей дозволяє
більш повно враховувати вплив різних факторів на перебіг досліджуваних
процесів.
Ефективність статистичного контролю підвищується засобами
механізації і автоматизації процесу контролю, застосуванням обчислювальної
техніки для обробки результатів вимірювань та їх перетворення в керуючі дії.
Залежно від умов розподіл похибок параметрів процесу, що
досліджується, може підлягати різним законам. У загальному випадку закон
розподілу описується залежністю:
Y=F(x),
де X - значення випадкової величини (наприклад розміру або відхилення
від номінального розміру);
Y - значення ординати кривої розподілу - густина ймовірності.
Вид кривої розподілу визначається числом і характером прояву факторів,
які мають вплив на досліджувану величину.
Наприклад, якщо при перебігу процесу, що досліджується, мають місце
тільки випадкові похибки, додатні і від'ємні значення яких відносно
номінального значення досліджуваної величини рівноймовірні (розподіл
Гауса), крива розсіювання приймає симетричну форму (рисунок 1).
Систематична (постійна) похибка форму кривої розсіювання не змінює, а
лише зміщує положення кривої в напрямку осі абсцис (рисунок 2, а).
Якщо на параметр, що вимірюється, впливає систематична похибка, яка
лінійно зростає, то розподіл відбувається по закону рівної ймовірності (рисунок
2,б).
Рисунок 1 - Крива нормального розподілу Гауса:
х - розмір, що вимірюється; у - емпірична частота; 8 - середнє
квадратичне відхилення
Рисунок 2 - Криві розподілу при дії таких факторів:
а) - систематичної похибки; б) - систематичної похибки, що лінійно
зростає; в) - закономірної похибки, яка зростає спочатку з уповільненням, а
пізніше з прискоренням; г) - температурних деформацій технологічної системи.
Якщо на параметр, що вимірюється, впливає похибка, яка закономірно
змінюється і зростає спочатку повільно, а пізніше з прискоренням, наприклад,
при тривалості процесу за первинних умов (один ріжучий інструмент від його
заточування до затуплення), розподіл розмірів відбувається по закону
трикутника, або закону Сімпсона (рисунок 2,в).
Якщо на параметр, що вимірюється, впливає похибка, яка змінюється
закономірно (наприклад, викликана температурними деформаціями
технологічної системи), то крива розподілу розмірів приймає форму, приведену
на рисунку 2, г.
Як показує практика, розподіл параметру, що вимірюється, часто
відповідає нормальному закону розподілу - закону Гауса. В зв'язку з цим,
вказана методика розглядається стосовно аналізу похибок, які мають
нормальний розподіл.
На початку експериментального дослідження значну увагу слід приділяти
обґрунтуванню вибіркової сукупності об'єктів, що будуть піддаватись
оцінюванню та вимірюванню в ході експерименту.
Достовірне обґрунтування вибірки для експериментального дослідження
дозволяє визначити надійність підсумкових результатів та висновків стосовно
якісних або кількісних показників, що вивчаються статистичними методами.
Спираючись на наукові джерела теоретичного обґрунтування
математичних методів опрацювання експериментального дослідження
[Бурлачук Л. Ф. Словарь-справочник по психодиагностике / Л. Ф. Бурлачук,
С. М. Морозов. - К. : Наукова думка, 1989. - 200 с.; с. 40], ми для умов, у яких
проводиться експеримент, обсяг нормальної розподіленої вибірки (п), що безперервно змінюється, можемо обчислити за формулою:
mu = (((t*sigma)/Lambda) ^ 2)/(1 + 1/N * ((t*sigma)/Lambda) ^ 2)
де № обсяг генеральної
- значення абсциси для кривої нормального розподілу з бажаною точністю оцінювання (для ймовірності P = 0.95t = 1, 96 );
А рівень точності в долях від середнього арифметичного суми змінних
вибірки (для ймовірності P = 0, 9 Delta = 0.5 );
- стандартне відхилення Х (для ймовірності P = 0, 9 sigma = 2, 0 )
Вибірка це множина об'єктів, подій, зразків або сукупність вимірів, за допомогою визначеної процедури вибраних з статистичної популяції або генеральної сукупності для участі в дослідженні. Зазвичай, розміри популяції (генеральної сукупності) дуже великі, що робить прийняття до уваги всіх членів популяції непрактичним або неможливим. Вибірка представляє собою множину або сукупність певного обсягу, члени якої збираються і статистичні характеристики обчислюються таким чином, що в результаті можна зробити висновки або екстраполяцію із вибірки на всю популяцію або генеральну сукупність.
Побудова емпіричної кривої розподілу
Побудова емпіричної кривої розподілу похибок та гістограми
проводиться в такій послідовності:
По результатах вимірювань деталей вибірки визначається різниця між найбільшим і найменшим розмірами R (розмах вибірки): RXmax Xmin, (3.2)
де Хтах максимальний розмір, (мм);
Xmin мінімальний розмір, (мм).
Визначається ширина інтервалу
де ширина інтервалу.
(3.3)
a-Rf.
При обсязі вибірки n = 50 шт. число інтервалів рекомендується приймати рівним f = 6.8 . Для компенсації похибки вимірювань ширину інтервалу слід приймати приблизно в два рази більшою, ніж ціна поділки вимірювального
приладу. Величина в розбивається на інтервали шириною а
Підраховується частота", тобто кількість деталей, які попадають в кожен з інтервалів. При цьому, в кожен інтервал включають деталі з розмірами, які знаходяться в межах від найменшого значення інтервалу включно, до найбільшого значення інтервалу, не включаючи його. В останній інтервал
включають як найменше, так і найбільше значення. Визначають середини
X
інтервалів Xi (середні розміри інтервалів).
Результати розрахунків заносять в таблицю 3.1
Результати розрахунків заносяться в таблицю 3.1:

\begin{table}[H]
\centering
\caption{Вихідні дані для побудови кривої розподілу}
\begin{tabular}{|c|c|c|c|c|c|c|c|c|c|}
\hline
\multicolumn{2}{|c|}{Інтервали розмірів} & Емпірична частота & Середина інтервалу & Параметр & Теоретична частота & Теоретична частота округлена \\
\hline
від & до & $n_i$ & $X_i$ & $t$ & $Z$ & $n_{i\text{т}}$ & $n_{i\text{т.окр}}$ \\
\hline
\end{tabular}
\end{table}

Для побудови гістограми розподілу, взірець якої подано на рисунку, на осі абсцис відкладають інтервали розмірів і на кожному з цих інтервалів, як на основі, будують прямокутник, висота якого визначається відповідною емпіричною частотою. На верхніх сторонах цих прямокутників відкладають середні значення інтервалів і через отримані точки проводять криву. Отриманий таким чином графік називають полігоном розподілу.

Побудова теоретичної кривої нормального розподілу.

По зовнішньому вигляду цієї кривої можна наближено визначити закон розподілу похибок в генеральній сукупності. Для більш точного оцінювання необхідно зіставити емпіричну криву розподілу з теоретичною. З цією метою для кожного інтервалу значень необхідно обчислити теоретичні частоти і по них побудувати теоретичну криву розподілу.

Рівняння кривої нормального розподілу має вигляд:

\[
\varphi(x) = \frac{1}{\sigma \sqrt{2\pi}} \cdot e^{-\frac{(x - \bar{x}_0)^2}{2\sigma^2}} \tag{3.4}
\]

де $\varphi(x)$ — густина ймовірності (імовірність появи того чи іншого значення випадкової величини);

$\sigma$ — середнє квадратичне відхилення випадкової величини;

$\bar{x}_0$ — середнє значення випадкової величини;

$x$ — поточне її значення;

$e$ — основа натуральних логарифмів.

В експериментальних дослідженнях в якості наближених оцінок параметрів генеральної сукупності $\bar{x}_0$ та $\sigma$ використовують вибіркове середнє та вибіркове середнє квадратичне відхилення $S$, які обчислюються за формулами:

\[
\bar{x} = \frac{\sum_{i=1}^{k} x_i n_i}{n} \tag{3.5}
\]
\[
S = \sqrt{ \frac{ \sum\limits_{i=1}^{k} n_i (x_i - \bar{x})^2 }{n} }, \tag{3.6}
\]

При побудові теоретичної кривої нормального розподілу приймається, що $\bar{x}_0 = \bar{x}$ та $\sigma = S$.

Приблизно можна рахувати, що

\[
\varphi(x) = \frac{n_i^T}{n \cdot a} = \frac{1}{\sigma \sqrt{2 \pi}} e^{-\frac{(x - \bar{x}_0)^2}{2 \sigma^2}}, \tag{3.7}
\]

де $n_i^T$ — теоретична частота, 

$a$ — ширина інтервалу (величина $a$ введена в рівняння (4.7) приведення теоретичної кривої нормального розподілу до того масштабу, в якому викреслена емпірична крива).

З рівняння (4.7) будемо мати:

\[
n_i^T = \frac{n a}{\sigma} \cdot \frac{1}{\sqrt{2 \pi}} e^{-\frac{(x - \bar{x}_0)^2}{2 \sigma^2}}, \tag{3.8}
\]

Якщо у вираз (3.8) підставити $t = (x - \bar{x}_0) / \sigma$, то отримаємо:

\[
n_i^T = \frac{n a}{\sigma} \cdot \frac{1}{\sqrt{2 \pi}} e^{-\frac{t^2}{2}} = Z_i,
\]

Позначимо $\frac{1}{\sqrt{2 \pi}} e^{-\frac{t^2}{2}} = Z_i$, і при цьому, що $\sigma = S$.

Тоді формула (8) матиме вигляд:

\[
n_i^T = \frac{n \cdot a}{S} \cdot Z_i, \tag{3.9}
\]

Величина $Z_i$, обчислена для різних значень $t$, приведена в таблиці додатку А. Значення $t$ для кожного інтервалу розмірів знаходяться по формулі:

\[
t = \frac{|x - \bar{x}_i|}{S}, \tag{3.10}
\]

Таким чином, для підрахунку теоретичних частот необхідно для кожного інтервалу розмірів по формулі (10) визначити значення $t$, по таблиці додатку А знайти $Z_i$ і потім скористатися формулою (9). При підрахунку теоретичних частот доцільно користуватись таблицею (див. табл. А1). Графік теоретичної кривої нормального розподілу зазвичай співпадає з графіком емпіричної кривої (рисунок 3).
Необхідно відзначити, що теоретична крива нормального розподілу також може бути побудована по характеристичних точках. Координати характеристичних точок кривої нормального розподілу наведені в таблиці~2.

\begin{table}[H]
\centering
\caption{Координати точок нормального розподілу}
\begin{tabular}{|c|c|c|}
\hline
Характеристика точки & Абсциса & Ордината \\
\hline
Вершина кривої & $X$ & $n_i^T = 0.4 \cdot (n \cdot a) / S$ \\
\hline
Точка перегину & $X \pm S$ & $n_i^T = 0.24 \cdot (n \cdot a) / S$ \\
\hline
Характерна точка & $X \pm 2S$ & $n_i^T = 0.054 \cdot (n \cdot a) / S$ \\
\hline
Характерна точка & $X \pm 3S$ & $n_i^T = 0$ \\
\hline
\end{tabular}
\end{table}

Метою системи управління процесом є прийняття економічно вірних рішень відносно дій, які пов'язані з процесом. Процес діє тільки в статистично керованому стані, якщо джерелом мінливості є тільки постійні причини. Тому, однією із функцій системи управління процесом є подача статистичного сигналу в ситуаціях, коли з'являються випадкові (особливі) причини мінливості. Це дозволяє прийняти потрібні дії стосовно усунення цих випадкових причин.

Контрольні карти для кількісної ознаки особливо корисні згідно таких причин:

1. Більшість процесів і результати мають вимірювальні характеристики. Кількісні значення (приклад 16{,}45 мм) дають більше інформації, ніж просте ``так -- ні'' (наприклад, діаметр всередині допуску).

Хоч отримання одного даного, що було заміряне, дорожче, ніж отримання одного ``так -- ні''. Потрібно менше одиниць вимірювань, щоб отримати більше інформації про процес. Тому загальна ціна вимірювань в деяких випадках дешевша.

Для прийняття потрібного рішення, потрібно менше одиниць вимірювань, тобто менше часу затримки виготовлення деталі.

З кількісними даними може бути проаналізована настройка процесу і покращення може бути кількісно оцінене, якщо навіть всі значення лежать всередині допуску.

Контрольні карти можуть пояснити положення дані процесу стосовно розкиду і положення. Тому ці карти можуть бути аналізовані по напрямах: одна карта з даними щодо положення (середина процесу) і друга -- щодо розкиду.

Використання контрольних карт є процесом, в якій повторюються три основні фази: збір даних, управління та аналіз.

Збір: дані про характеристику процесу вивчаються і приводять до форми, в якій вони можуть бути нанесені на контрольну карту.
Керування: на основі даних розраховуються пробні контрольні границі.
Контрольні границі не є границями допуску, а основані на мінливості процесу.
Потім дані порівнюються з контрольними границями, щоб з'ясувати стабільний
процес чи ні.

Аналіз: після приведення процесу до статистично керованого стану
контрольна карта продовжується вестися для аналізу. Індекси відтворення
процесу також розраховуються.

Оцінка керованості процесом

Контрольні карти можуть бути пояснені наступним чином: якщо
мінливість процесу від деталі до деталі і середнє для процесу є постійними, то
окремі точки підпуть розмах $R$ і середнє $\overline{X}$ можуть змінюватися тільки
випадково, але при цьому вони рідко виходять за контрольні границі. Мета
використання контрольних карт – розпізнати ознаки того, що мінливості або
середнє значення не залишаються на постійному рівні, а вони вийшли із
керованого стану і необхідні відповідні дії. Карти $R$ та $\overline{X}$ аналізуються
окремо, але іноді їх порівняння можуть дати додаткову інформацію про
особливі причини, які впливають на процес.

Аналіз даних, які нанесені на карту.

Карта $R$ аналізується першою.

1. Крапки за межами контрольних границь.

\begin{figure}[H]
\centering
\includegraphics[width=0.6\linewidth]{chart1.png} % тут вставити шлях до вашого зображення
\caption{Рис. 3.1 Карта 1}
\label{fig:chart1}
\end{figure}

Наявність однієї або декількох крапок за межами контрольних границь –
перша ознака відсутності керованості в цій карті. Оскільки крапки за межами
контрольних границь дуже рідкі, якщо присутні тільки звичайні причини
мінливості, то можливо передбачити, що на картку діє особлива причина.
Тобто, кожна крапка за межею контрольної границі є сигналом для аналізу
процесу на наявність особливої причини.

2. Крапка вище контрольної границі для карти $R$ (карта розмахів) є
ознакою одного із змін:

\begin{itemize}
    \item неправильні розрахунки контрольних границь або нанесення точок;
    \item збільшення мінливості від деталі до деталі або розкид розподілення збільшується;
    \item змінено вимірювальну систему (поміняно набір або індикатор);
    \item виявлено особливу причину (вибірка з відбірки $n$ більше) є ознакою однієї із причин:
    \begin{itemize}
        \item контрольна границя або нанесення картки помилкові;
        \item розподілу зменшились (тобто стала вужчими);
    \end{itemize}
\end{itemize}

\begin{itemize}
    \item[в)] вимірювальна система змінилася.
    \item[4.] Невпорядковане розташування крапок.\\
    Невпорядковане розташування крапок, навіть якщо вони знаходяться в контрольних границях, може свідчити про некерованість процесом.
    \begin{itemize}
        \item[а)] процес некерований по розмаху (серія крапок вище та нижче середньої лінії);
    \end{itemize}
\end{itemize}

\begin{figure}[H]
\centering
\includegraphics[width=0.6\linewidth]{chart2.png} % вставте шлях до вашого зображення
\caption{Рис. 3.2. Карта 2}
\label{fig:chart2}
\end{figure}

\begin{itemize}
    \item[б)] процес некерований по розмаху (зростаюча серія крапок)
\end{itemize}

Кожна із наступних ознак є причиною того, що почався зсув процесу:\\
а) 7 крапок підряд по одну чи іншу сторону процесу\\
б) 7 крапок підряд послідовно зростають або послідовно падають.

Серія розмахів вище середнього значення або зростаюча серія означають:

\begin{figure}[H]
\centering
\includegraphics[width=0.6\linewidth]{chart3.png} % вставте шлях до вашого зображення
\caption{Рис. 3.3 Серія розмахів}
\label{fig:chart3}
\end{figure}

\begin{itemize}
    \item[а)] зріс розкид вихідних значень, які могли виникнути із-за нерегулярної причини (неправильного обладнання його фізичного зносу);
    \item[б)] пройшли зміни в вимірювальній системі (наприклад, новий калібр або індикатор).
\end{itemize}

Серія розмахів нижче середньої лінії або спадаюча серія крапок означає:\\
а) зменшився розкид вихідних значень, що є хорошою обставиною, яку потрібно вивчити для широкого використання;\\
б) пройшла зміна в вимірювальній системі, яка приховує всі зміни;\\
г) незвичайна поведінка процесу.

Відстань крапок від $R$ (середньої лінії): в основному біля 2/3 нанесених крапок повинні лежати всередині 1/3 полоси між контрольними границями.

Процес некерований (крапки дуже близько до середньої лінії).

\begin{figure}[H]
\centering
\includegraphics[width=0.6\linewidth]{chart4.png} % замініть на вашу картинку
\caption{Рис. 3.4 Приклад некерованого процесу}
\label{fig:chart4}
\end{figure}

Процес некерований (крапки дуже близько до контрольних границь):

\begin{figure}[H]
\centering
\includegraphics[width=0.6\linewidth]{chart5.png} % замініть на вашу картинку
\caption{Рис. 3.5 Приклад Некерованого процесу 2}
\label{fig:chart5}
\end{figure}

Якщо більше ніж 2/3 нанесених крапок (90\%) лежать близько до $\bar{R}$ (середньої лінії), то контрольні границі або нанесені крапки мають помилки.

Якщо 2/3 нанесених крапок лежать близько контрольних границь, то:

а) контрольні границі або нанесені крапки мають помилки;

б) процес або метод взяття виборок роз’ється: кожна підгрупа систематично показує два або більше число потоків.

Карта $\bar{X}$ аналізується аналогічно карті $R$.

Порядок виконання роботи

1. Ознайомитись із теоретичними відомостями.

2. Обчислити обсяг нормальної розподіленої вибірки (n).

3. Відповідно до вказівок викладача провести вимірювання деталей вибірки. Результати вимірювань занести в протокол.

4. Визначити значення між найбільшим та найменшим розмірами деталей у виборці (розмах вибірки). Розділити $R$ на $f = 6 \ldots 8$ інтервалів. Знайти ширину інтервалу $a = R / f$.

5. Визначити середини інтервалів $X_i$. Підрахувати частоту для кожного інтервалу. Результати підрахунків занести в табл.1.

6. Побудувати гістограму та емпіричну криву розподілу розмірів (рисунок 3). Масштаб по осі абсцис прийняти таким, щоб висота емпіричної кривої складала 0.6\ldots 0.7 від її довжини.

Пояснення рисунка 3 — Гістограма (1), емпірична крива (2); крива нормального розподілу розмірів має бути побудована (3).

7. Згідно формул (5) та (6) підрахувати вибіркове середнє $X$ та вибіркове середнє квадратичне відхилення $S$ розмірів. Користуючись

формулами (9) і (10) та таблицею додатку A, визначити теоретичні частоти $n_t$ нормального розподілу для кожного інтервалу. Результати підрахунків занести в табл.1. Накреслити графік теоретичної кривої нормального розподілу, сумістивши його з графіком з емпіричною кривою. Занести дані в контрольну карту.

Традиційно карти $\overline{X}$ та $R$ будуються одна на другій: карти $\overline{X}$ над картою $R$ і далі йде блок даних. Значення $\overline{X}$ та $R$ відкладаються на вертикальних осях. Номери підгруп відкладаються на горизонтальній осі. Карта також містить блок даних, що передбачає місця для кожного результату вимірювання, а також для середніх результатів вимірювання, середніх розмахів і дати / часу чи іншої ідентифікації кожної підгрупи, які повинні повністю заповнюватися.

Розрахувати середнє значення і розмах для кожної підгрупи.

Характеристики, що наносяться на карту — це середні вибірок $\overline{X}$ та розмах вибірок $R$ для кожної підгрупи. Вони показують поведінку середнього для всього процесу і його змінюваність відповідно.

Для кожної підгрупи обчислити

\[
\overline{X} = \frac{X_1 + X_2 + \ldots + X_n}{n}, \tag{3.11}
\]

\[
R = X_{max} - X_{min},
\]

де $X_1, X_2, \ldots$ — індивідуальні значення в підгрупі;

$n$ — об'єм вибірки для підгрупи.

Вибрати шкали для контрольних карт.

Вертикальні шкали карт призначені для значень $\overline{X}$ та $R$ відповідно. Корисні деякі вказівки по визначенню шкал, хоча вони можуть змінюватися в конкретних умовах. Для карти $\overline{X}$ різниця між верхнім та нижнім краями шкали повинна бути, в крайньому разі, вдвічі більша різниці між найбільшим та найменшим значеннями середніх підгруп $\overline{X}$. Для карти $R$ ця шкала повинна мати значення від нуля до двократного найбільшого розмаху $R$, що спостерігається в процесі.

Нанести середні значення та розмахи на контрольні карти.

Нанесення середніх значень та розмахів на відповідні карти. Це робиться у звичайному порядку шкал. З'єднати точки лініями, щоб візуально знайти хід процесу.

Якщо проявиться неналежний хід точки, перевірити їх розумність. Якщо деякі точки неправильні, їх треба пояснити. Неправильні вимірювання слід видалити і виміряти знову. В іншому разі, не нанесення точки відповідних $\overline{X}$ та $R$ значень на карту є неприпустимим.

Розрахувати центральну лінію (середній розмах і середнє для процесу).

\[
\overline{R} = \frac{R_1 + R_2 + \ldots + R_k}{k}, \tag{3.12}
\]

\[
\overline{X} = \frac{X_1 + X_2 + \ldots + X_n}{k}, \tag{3.13}
\]

де $k$ - число підгруп,

$R_1$ і $\overline{X}_1$ - розмах та середнє першої підгрупи,

$R_2$ і $\overline{X}_2$ - те ж саме для другої підгрупи і т. д.

Розрахувати контрольні границі.

Контрольні границі розраховують, щоб визначити, наскільки середні та розмахи підгруп можуть змінюватися, якщо присутні лише звичайні причини мінливості. Вони основані на об'ємі підгруп і величині мінливості всередині підгруп, що відображається розмахами.

Обчислити верхню та нижню контрольні границі для розмахів і середніх по формулах:

\[
UCL_R = D_4 \overline{R}
\]
\[
LCL_R = D_3 \overline{R}
\]
\[
UCL_{\overline{X}} = \overline{X} + A_2 \overline{R}
\]
\[
LCL_{\overline{X}} = \overline{X} - A_2 \overline{R}
\]

де $D_4$, $D_3$, $A_2$ - константи, які залежать від об'єму підгрупи, для $n$ від 2 до 10 константи приведені в наступній таблиці, взятій з додатка.

\begin{center}
\begin{tabular}{|c|c|c|c|c|c|c|c|c|c|c|}
\hline
$n$ & 2 & 3 & 4 & 5 & 6 & 7 & 8 & 9 & 10 \\
\hline
$A_2$ & .27 & .57 & .28 & .11 & .10 & .09 & .08 & .08 & .07 \\
\hline
$D_3$ & 0 & .08 & .14 & .18 & .22 & .26 & .28 & .29 & .31 \\
\hline
$D_4$ & .88 & .82 & .73 & .58 & .48 & .42 & .37 & .34 & .31 \\
\hline
\end{tabular}
\end{center}

(Для об'ємів вибірки менше 7, значення $LCL_R$ від'ємне. В таких випадках $LCL$ не будується. Це означає, що для підгрупи з 6 одиниць отримання 6 ``ідентичних'' вимірювань не буде являтися ознакою незвичайності).

Провести лінії середніх та контрольних границь на картах.

Провести суцільні горизонтальні лінії для середнього розмаху ($\overline{R}$) та середнього процесу ($\overline{X}$). Контрольні границі ($UCL_R$, $LCL_R$, $UCL_{\overline{X}}$, $LCL_{\overline{X}}$) провести штриховими горизонтальними лініями. Позначити найменування ліній. На період початкового обстеження вони розглядаються як пробні контрольні границі.


Add data for task:
Проведено вимірювання 30 деталей, значення діаметра (мм):

24.6, 24.7, 24.6, 24.8, 24.7, 24.6, 24.9, 24.5, 24.8, 24.7, 24.6, 24.6, 24.7, 24.5, 24.7, 24.8, 24.9, 24.7, 24.8, 24.6, 24.5, 24.6, 24.8, 24.9, 24.7, 24.7, 24.6, 24.6, 24.5, 24.7

Номінал: 24.7 мм
Граничні відхилення: ±0.2 мм (тобто припустимий діапазон: 24.5 – 24.9 мм)
Кількість підгруп: 6
Кількість елементів у підгрупі: 5

Can you provide a step-by-step solution to this problem, explaining how and why each step is taken? I'd also like to understand the rule concepts in more detail, including any common mistakes to avoid. Additionally, could you provide a similar example problem with a solution for further practice?
