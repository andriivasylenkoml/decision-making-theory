\noindent Лабораторна робота №6. Ручний та автоматизований симплекс-метод розв’язку оптимізаційних задач

\medskip

\noindent Мета роботи: навчитись застосовувати симплекс-метод для вирішення оптимізаційних задач виробництва. Застосування надбудови «Пошук рішення» у Excel.

\medskip

\noindent Завдання роботи: Визначити максимально ефективний план випуску продукції на виробництві за умов обмежень у сировині та за умови отримання максимального прибутку.

\medskip

\noindent Теоретичні відомості

\noindent Нехай задача ЛП має оптимальний розв’язок. З геометричної точки зору це означає, що існує вершина багатокутника розв’язків, де лінійна функція досягає оптимального значення. Кожній вершині багатокутника відповідає опорний план. А кожний опорний план визначається системою $m$ лінійно незалежних векторів, що містяться серед $n$ векторів $A_1$, $A_2$, ..., $A_n$ системи обмежень. Щоб знайти оптимальний план, досить розглянути лише опорні плани. Їх кількість не перевищує $C_n^m$. Для великих значень $n$ і $m$ знайти серед них оптимальний простим перебором дуже важко. Тому необхідно мати такий аналітичний метод, що дає можливість цілеспрямовано здійснювати перебір опорних планів. Таким методом є \textit{симплексний метод}. Виходячи з одного (початкового) опорного плану, симплексний метод забезпечує побудову нового опорного плану, що надає лінійній функції менші значення у порівнянні з попереднім планом. Цей процес продовжується поки не буде знайдено оптимальний план.

\noindent Оскільки кількість опорних планів обмежено, то обмежено і кількість розв’язків. Відзначимо ще на розв’язку, що симплекс-метод встановлює цей факт у ході розв’язку задачі. Це означає, що не під час обчислень можна встановити, чи є система обмежень сумісною і чи є лінійна функція обмеженого на множині планів задачі лінійного програмування.

\noindent Отже, симплекс-метод дає спосіб обчислення опорного плану, перевіряє побудований опорний план на оптимальність і надає спосіб побудови наступного опорного плану, що буде ближче до оптимального. Завдяки цьому симплекс-метод полягає в послідовному поліпшенні плану і тому його називають ще методом послідовного поліпшення плану.

\noindent Розв’язок задач симплексним методом складається з двох етапів: знаходження початкового опорного плану і знаходження оптимального плану. При цьому алгоритм симплексного методу застосовний лише до канонічної форми запису задачі ЛП. Тому, перед тим як розв’язувати задачу, систему обмежень необхідно спочатку привести до канонічної форми.

\medskip

\noindent Застосування симплекс-таблиць

\noindent Оскільки базис системи — одиничний, то коефіцієнти у виразі вектора через базисні будуть його компоненти, тобто $x_j = a_{ij}$ $(i=1,2,...,m;\; j=1,2,...,n)$.

\noindent \textit{п).} Обчислення наступного опорного плану та перевірку його оптимальності зручно виконувати, записавши умову задачі і далі, знайдені після побудови початкового опорного плану у симплексній таблиці. У першому її стовпці записано номери рядків таблиці. Стовпець базису містить базисні вектори. У стовпці $C$ базису записано коефіцієнти даної лінійної функції, які відповідають базисним векторам. У стовпці $A_0$ - початковий опорний план. У цьому ж стовпці з результатів обчислень знаходять оптимальний план.

Стовпці $A(j = 1,2,\ldots,n)$ заповнено коефіцієнтами вектора $A_j$ через базисні вектори. У верхньому рядку таблиці записано коефіцієнти цільової функції $F(X)$, яка вона набуває при початковому опорному плані, а в стовпцях $A_j$ - значення коефіцієнтів при відповідних змінних. В останньому рядку таблиці, стовпці $A_j$ для відповідних елементів містять оцінки векторів $A_j$. Оцінки обчислюються за спеціальною схемою, що базується на визначенні елементів $C$ базису відповідно до таким чином:

1. Розглянути оцінку плану $(m+1)$-го рядка. Якщо всі оцінки не додатні, то опорний план $X_0$ оптимальний і мінімум лінійної функції дорівнює $F(X_0)$. Якщо серед оцінок є хоча б одна додатна, то перейти до п.2.

2. Якщо хоча б один додатній оцінка відповідає $j$-су елементу стовпця $A_j$ не додатні, то це означає, що лінійна функція на многограннику розв’язків не обмежена знизу і задача не має оптимального розв’язку. Якщо в кожному стовпці $A_j$ з додатними оцінками є хоча б одна додатня коефіцієнт, то перейти до п.3.

\textit{Коментар.} Якщо у кожному стовпці $A_j$, що відповідають додатним оцінкам $m_{i,j}$, є додатні коефіцієнти, то план $X_0$ оптимальний і немає можливості побудувати новий опорний план, який надасть лінійній функції значення, менше мінімуму. Задача немає розв’язку. Якщо хоча б у 1 стовпці є хоча б один додатній елемент, вибирають цей стовпець $A_j$, в якому оцінка найбільша. Вектор, який вводять у базис, позначають в п.3, як вектор, який замінює новим вектором, який входить у базис, вибирають в п.3, як вектор, який замінює новим.

3. Вибрати рядок з найбільшою додатною оцінкою. Нехай це буде вектор $A_\theta$. Перейти до п.4.

4. Зі співвідношення вибрати елемент $\theta_{ij}$ в стовпці $A_j$ і рядку $i$, по якому вводиться у базис вектор $A_j$. Результат записати у відповідній клітинці нового плану таблиці. Перейти до п.5.

5. Вибрати з рядка $A_j$ додатній елемент у той, який відповідає найменшому значенню частки по п.4. Нехай це буде вектор $A_\theta$. Перейти до п.6.

\textit{Коментар.} Елемент, що вводиться у базис у відповідному елементі, по якому він визначається, називається ведучим. Після визначення ведучого елемента, інші елементи підлягають заміні. При цьому в стовпці нового таблиці у новий план буде складено тільки рядки з номером 1. У цьому ж плані будуть усі нові результати з ведучим елементом у клітинці нового плану таблиці.

6. Поділити елементи $i$-го рядка, які відповідають векторам $A(j=1,2,\ldots,n)$, на ведучий елемент і результати записати у відповідній клітинці $i$-го рядка нової таблиці. Перейти до п.7.

